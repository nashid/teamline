%\documentclass[journal]{vgtc}                % final (journal style)
% \documentclass[review,journal]{vgtc}         % review (journal style)
%\documentclass[widereview]{vgtc}             % wide-spaced review
\documentclass[preprint,journal]{vgtc}       % preprint (journal style)

%% Uncomment one of the lines above depending on where your paper is
%% in the conference process. ``review'' and ``widereview'' are for review
%% submission, ``preprint'' is for pre-publication, and the final version
%% doesn't use a specific qualifier.

%% Please use one of the ``review'' options in combination with the
%% assigned online id (see below) ONLY if your paper uses a double blind
%% review process. Some conferences, like IEEE Vis and InfoVis, have NOT
%% in the past.

%% Please note that the use of figures other than the optional teaser is not permitted on the first page
%% of the journal version.  Figures should begin on the second page and be
%% in CMYK or Grey scale format, otherwise, colour shifting may occur
%% during the printing process.  Papers submitted with figures other than the optional teaser on the
%% first page will be refused. Also, the teaser figure should only have the
%% width of the abstract as the template enforces it.

%% These few lines make a distinction between latex and pdflatex calls and they
%% bring in essential packages for graphics and font handling.
%% Note that due to the \DeclareGraphicsExtensions{} call it is no longer necessary
%% to provide the the path and extension of a graphics file:
%% \includegraphics{diamondrule} is completely sufficient.
%%
\ifpdf%                                % if we use pdflatex
  \pdfoutput=1\relax                   % create PDFs from pdfLaTeX
  \pdfcompresslevel=9                  % PDF Compression
  \pdfoptionpdfminorversion=7          % create PDF 1.7
  \ExecuteOptions{pdftex}
  \usepackage{graphicx}                % allow us to embed graphics files
  \DeclareGraphicsExtensions{.pdf,.png,.jpg,.jpeg} % for pdflatex we expect .pdf, .png, or .jpg files
\else%                                 % else we use pure latex
  \ExecuteOptions{dvips}
  \usepackage{graphicx}                % allow us to embed graphics files
  \DeclareGraphicsExtensions{.eps}     % for pure latex we expect eps files
\fi%

%% it is recomended to use ``\autoref{sec:bla}'' instead of ``Fig.~\ref{sec:bla}''
\graphicspath{{figures/}{pictures/}{images/}{./}} % where to search for the images

\usepackage{microtype}                 % use micro-typography (slightly more compact, better to read)
\PassOptionsToPackage{warn}{textcomp}  % to address font issues with \textrightarrow
\usepackage{textcomp}                  % use better special symbols
\usepackage{mathptmx}                  % use matching math font
\usepackage{times}                     % we use Times as the main font
\renewcommand*\ttdefault{txtt}         % a nicer typewriter font
\usepackage{cite}                      % needed to automatically sort the references
% \usepackage{tabu}                      % only used for the table example
% \usepackage{booktabs}                  % only used for the table example
%% We encourage the use of mathptmx for consistent usage of times font
%% throughout the proceedings. However, if you encounter conflicts
%% with other math-related packages, you may want to disable it.
\usepackage[para]{threeparttable}


%% In preprint mode you may define your own headline.
\preprinttext{}

%% If you are submitting a paper to a conference for review with a double
%% blind reviewing process, please replace the value ``0'' below with your
%% OnlineID. Otherwise, you may safely leave it at ``0''.
\onlineid{0}

%% declare the category of your paper, only shown in review mode
\vgtccategory{Research}
%% please declare the paper type of your paper to help reviewers, only shown in review mode
%% choices:
%% * algorithm/technique
%% * application/design study
%% * evaluation
%% * system
%% * theory/model
\vgtcpapertype{application/design study}

%% Paper title.
\title{Teamline: Visualizing small team code contributions \\ \large CPSC 547 Project Proposal}

%% This is how authors are specified in the journal style

%% indicate IEEE Member or Student Member in form indicated below
\author{
  Nick Bradley\\   \texttt{nbrad11@cs.ubc.ca}
  \and
  Felix Grund\\    \texttt{ataraxie@cs.ubc.ca}
}
% \authorfooter{
% %% insert punctuation at end of each item
% \item
%  Nick Bradley. Software Practices Lab, Department of Computer Science, University of British Columbia. E-mail: nbrad11@cs.ubc.ca.
% \item
%  Felix Grund. Software Practices Lab, Department of Computer Science, University of British Columbia. E-mail: ataraxtie@cs.ubc.ca.
% }

%other entries to be set up for journal
%\shortauthortitle{Biv \MakeLowercase{\textit{et al.}}: Global Illumination for Fun and Profit}
%\shortauthortitle{Firstauthor \MakeLowercase{\textit{et al.}}: Paper Title}

%% Abstract section.
% \abstract{Duis autem vel eum iriure dolor in hendrerit in vulputate
% velit esse molestie consequat, vel illum dolore eu feugiat nulla
% facilisis at vero eros et accumsan et iusto odio dignissim qui blandit
% praesent luptatum zzril delenit augue duis dolore te feugait nulla
% facilisi. Lorem ipsum dolor sit amet, consectetuer adipiscing elit,
% sed diam nonummy nibh euismod tincidunt ut laoreet dolore magna
% aliquam erat volutpat. Ut wisi enim ad minim veniam, quis nostrud exerci tation ullamcorper
% suscipit lobortis nisl ut aliquip ex ea commodo consequat. Duis autem
% vel eum iriure dolor in hendrerit in vulputate velit esse molestie
% consequat, vel illum dolore eu feugiat nulla facilisis at vero eros et
% accumsan et iusto odio dignissim qui blandit praesent luptatum zzril
% delenit augue duis dolore te feugait nulla facilisi.%
% } % end of abstract

%% Keywords that describe your work. Will show as 'Index Terms' in journal
%% please capitalize first letter and insert punctuation after last keyword
% \keywords{Radiosity, global illumination, constant time}

%% ACM Computing Classification System (CCS).
%% See <http://www.acm.org/class/1998/> for details.
%% The ``\CCScat'' command takes four arguments.

% \CCScatlist{ % not used in journal version
%  \CCScat{K.6.1}{Management of Computing and Information Systems}%
% {Project and People Management}{Life Cycle};
%  \CCScat{K.7.m}{The Computing Profession}{Miscellaneous}{Ethics}
% }

%% Uncomment below to include a teaser figure.
% \teaser{
%   \centering
%   \includegraphics[width=\linewidth]{CypressView}
%   \caption{In the Clouds: Vancouver from Cypress Mountain. Note that the teaser may not be wider than the abstract block.}
% 	\label{fig:teaser}
% }

%% Uncomment below to disable the manuscript note
%\renewcommand{\manuscriptnotetxt}{}

%% Copyright space is enabled by default as required by guidelines.
%% It is disabled by the 'review' option or via the following command:
% \nocopyrightspace

% \vgtcinsertpkg

%%%%%%%%%%%%%%%%%%%%%%%%%%%%%%%%%%%%%%%%%%%%%%%%%%%%%%%%%%%%%%%%
%%%%%%%%%%%%%%%%%%%%%% START OF THE PAPER %%%%%%%%%%%%%%%%%%%%%%
%%%%%%%%%%%%%%%%%%%%%%%%%%%%%%%%%%%%%%%%%%%%%%%%%%%%%%%%%%%%%%%%%

\begin{document}

%% The ``\maketitle'' command must be the first command after the
%% ``\begin{document}'' command. It prepares and prints the title block.

%% the only exception to this rule is the \firstsection command
\firstsection{Introduction}

\maketitle

%% \section{Introduction} %for journal use above \firstsection{..} instead

Our tool visualizes data collected from AutoTest\footnote{http://github.com/nickbradley/autotest},
an automatic grading service used to grade code submissions for students in
CPSC310. The course is structured around a term-long coding project that is
divided into 5 deliverables/sprints, only the first 3 of which are graded by a
combination of AutoTest and TA, completed by teams consisting of 2-3 students.
The teams manage their shared code on GitHub\footnote{http://github.com}
using a basic git workflow: students pull the latest code changes from GitHub,
commit their modified code locally and then push those commits to GitHub for
other members to see. Every time a student pushes their changes, AutoTest is
automatically invoked and runs a private suite of tests against the modified
code. Results are stored in a NoSQL database with each record corresponding to a
single submission (push event). The relevant attributes are briefly described in
Table 1. We have collected data for over 24,000 submissions for the first two
deliverables; complete data for the third deliverable will be available on March
13. There are 285 students in 139 teams.

After a submission deadline, TAs meet with their assigned
teams to conduct a retrospective to discuss any challenges that arose
during the sprint and to ensure that the work was equitably distributed among the
team members. This typically consists of a TA asking some questions designed to
gauge a student's comprehension of the task and code. We may go so far as to
explicitly and privately ask each student how evenly they felt the workload was
split. The TAs assign a scaling factor to the deliverable grade. For example, if
the team got 90\% on the deliverable but one member did most of the work, the final
grades might be 90\%*1.0 = 90\% and 90\%*0.6 = 54\%. Unfortunately, it can be hard
to determine how much work was done by each student from these conversations since
the team member who contributed very little will attempt to spoof the TA while the
hard-working one may not want to rat out their partner. One possible solution is
to look at the commit history on GitHub to determine how many commits each student
made. This can be a decent proxy but can be misleading since different people have
different commit habits (some will commit every line, others only large changes)
and they may not reflect the actual contribution to the grade (i.e. commits that
don't directly increase the grade).

% \begin{table}[tb]
% \caption{Dataset Attributes}
% \label{tab:attributes}
% \centering
% \begin{tabular}{l l l}
% \hline
% \textbf{Attribute Name} & \textbf{Attribute Type} & \textbf{Description} \\
% \hline
% testGrade      & Quantitative & Percentage of private tests that passed against student code. \\
% coverageGrade  & Quantitative & Percentage of code executed by student written tests. \\
% finalGrade     & Quantitative & Computed as 0.8*testGrade + 0.2*coverageGrade. \\
% timestamp      & Sequential   & Unix time of record creation. \\
% commitSha      & Categorical  & The (partial) SHA-1 hash of the submitted commit. \\
% committer      & Categorical\tablefootnote{285 values currently; max <1000}  & The GitHub ID of the student making the submission. \\
% team           & Categorical\tablefootnote{139 values currently; max <1000}  & The team number, stored as \textit{teamXX}, where \textit{XXX} is a number between 2 and 199. \\
% deliverable    & Sequential   & The submission deliverable, which can have values \textit{d1}, \textit{d2}, or \textit{d3}.
% \hline
% \end{tabular}
% \end{table}

\begin{table*}[t]
  \label{tab:attributes}
  \centering
  \begin{threeparttable}
      \caption{Dataset Attributes.}
  \begin{tabular*}{\textwidth}{lll}
    \hline
    \textbf{Attribute Name} & \textbf{Attribute Type} & \textbf{Description} \\
    \hline
    testGrade      & Quantitative & Percentage of private tests that passed against student code. \\
    coverageGrade  & Quantitative & Percentage of code executed by student written tests. \\
    finalGrade     & Quantitative & Computed as 0.8*testGrade + 0.2*coverageGrade. \\
    timestamp      & Sequential   & Unix time of record creation. \\
    commitSha      & Categorical  & The (partial) SHA-1 hash of the submitted commit. \\
    committer      & Categorical\tnote{a}  & The GitHub ID of the student making the submission. \\
    team           & Categorical\tnote{b}  & The team number, stored as \textit{teamXX}, where \textit{XXX} is a number between 2 and 199. \\
    deliverable    & Sequential   & The submission deliverable, which can have values \textit{d1}, \textit{d2}, or \textit{d3}. \\
    \hline
  \end{tabular*}
  \begin{tablenotes}\footnotesize
    \item [a] 285 values currently; max $<1000$.
    \item [b] 139 values currently; max $<1000$.
\end{tablenotes}
\end{threeparttable}
\end{table*}

\section{Proposed Solution}
Our solution is to create a derived quantitative attribute \textit{commitContribution} that describes
the impact of a submission on the overall grade. In particular, the attribute is
the difference between the current submission and the previously graded one. We
visualize this along with other code metrics (Fig. 1) to gain a more
complete understanding of each student's contribution: those who made more grade-improving
submissions should receive a higher retrospective grade. Note that this visualization
is designed to assist the TAs in
making a judgment when assigning a grade and cannot replace them since students may
have chosen a different way to divide the work among team members.

Here we use the what-why-how framework \cite{Munzner:2014} to abstract our solution to the vis domain.

\textbf{What}. Table of graded submission records (items) with the attributes
described in Table 1. The dataset is static once it has been loaded on the page
but is dynamic in that the dataset grows with each new submission.

\textbf{Why}.Compare the contributions of team members and derive a retrospective
grade for each of them.

\textbf{How}. Our vis includes many encodings and idioms:
\begin{itemize}
  \item Separate individual submissions and align them on two (or three)
        parallel axes by timestamp.
  \item Submissions are represented by area marks. Marks that overlap are collapsed into
    a single point mark whose size encodes the number of submissions that were collapsed.
  \item Marks are coloured with luminance encoding the overall grade for the submission (max luminance for a grade of 100%).
    Note that the overall this is relatively unimportant compared to the ability to
    accurately compare the number and effect of submissions. This information is also available
    by hovering over the point mark. For these reasons, we are okay using an encoding
    that is of low effectiveness and may not be visible on all marks.
  \item Key data is always visible while interaction allows details about a submission to
    be seen:
    \begin{itemize}
      \item Hovering over a single mark displays a popup with detailed information about the submission.
      \item Clicking a single points opens a new browser tab that displays the corresponding commit on GitHub.
      \item Hovering over a grouped mark expands it to show all submissions it includes. Once expanded,
        the above interactions are allowed.
    \end{itemize}
\end{itemize}

\section{Scenario}
Imagine you are a TA tasked with scaling the final grade of each team member by
the amount they contributed. Upon meeting the team, you open their Teamline to
get a sense of the team dynamics: did they start early? did they work consistently?
what was their final grade? This information is immediately available to you
because Teamline defaults to showing the team-view of the most recently due deliverable.
Within the view, the you find that the team made a few early submissions that
increased their grade and then made a large number of submissions very close to
the deadline. From the navigation pane, you notice that one of the team members
contributed significantly to final grade unlike the other.

You then proceed to discuss with each team member individually about their contributions
to the project. The one who made the larger contribution according to Teamline, Bob,
had a clear understanding of the code and was able to discuss a couple of challenges
he encountered. At the end of the retrospective, Bob said that each of them had
done the work they agreed to do (which they thought was an even split) but that
his partner started very late which made him apprehensive.
To confirm this, you expand the team-view to show individual submissions made by
each team member and do in fact notice that all of the submissions for the other
member, Joe, were made the night before the due date and that Bobs submissions were
made earlier and more consistently.

Next, you talk to Joe. After he explains what he contributed, you mention that by
starting so late, it may negatively impact the group dynamic. Joe refutes this by claiming
that he started days earlier but, after you show him Teamline, agrees that he should
start earlier next time. While looking at Teamline, you also notice that Bob did
quite a bit more work for the previous deliverable as well. You point this out to
both Bob and Joe and they are a bit surprised. You help them divide up the work
for the next deliverable more equitably.

Later in the week, you decide to check how the team is proceeding. You immediately
see that several submissions were made. Curious to see if your discussion helped,
you expand the team-view and see that Joe has already made several submissions.
You feel much more confident having scaled back Joe's grade by only 20% since he
is now contributing more.


\section{Implementation Approach}
We have decided to implement Teamline as a web application. We decided on this for
a variety of reasons: our familiarity with web technologies, platform independence,
increase likelihood of adoption in CPSC310 (and other courses that will be using AutoTest),
and integration with other service used in the course. Teamline is minimally dependent
on existing AutoTest infrastructure, only requiring access to the database via a
REST endpoint, and is completely novel of the existing system including the dashboard.

Given the above, we will use HTML5, CSS3 and jQuery to implement our vis. With the
advances in CSS, we believe we will not need a vis-specific library like D3.js but
we will finalize this decision once we have started coding.

\section{Expertise}
We decided on this project, in part, because we are both currently TAs for CPSC310
and have experienced the challenges of assigning a fair retrospective grade. In
addition, we are both excited to make use of the otherwise largely unused data.

In addition, Nick wrote and is currently managing the AutoTest system including
the database. As such, he has a total understanding of the data: how it was created,
its limitations, and some ways it can be meaningfully linked with other data sources
like GitHub.

Finally, this project is mildly interesting from a research perspective since it
is in our research area of software engineering. At a high level, it will be
interesting to see how software engineering students use the git workflow to
manage their project and collaborate with their team members.

\section{Milestones and Schedule}
We are prepared to spend about 108 hours towards this project. Table \ref{tab:schedule} provides
a breakdown of the project's tasks.

\begin{table*}[t]
  \caption{Task Schedule.}
  \label{tab:schedule}
  \centering
  \begin{tabular*}{\textwidth}{lrll}%{l r l l}@{\extracolsep{\fill}}
    \hline
    \textbf{Task} & \textbf{Est Time} & \textbf{Deadline} & \textbf{Description} \\
    \hline
    Pitch (x2)                       &  8 & Feb. 16  & Create slides, rehearse pitch. \\
    Proposal                         & 12 & Mar. 6   & Discuss project ideas, create mockups, write proposal. \\
    Project Review 1                 &  2 & Mar. 21  & Prepare slides. \\
    Interim writeup                  &  6 & Mar. 31  & Summary of progress, completed previous work section. \\
    Project Review 2                 &  2 & Apr. 2   & Prepare slides, have some version of demo ready. \\
    Implementation                   & 48 & Apr. 7   & Completed vis tool. \\
     - Create database view          &  4 & Mar. 14  & Create view(s) of computed/derived attributes in CouchDB. \\
     - Create tabs/buttons           &  4 & Mar. 21  & Set up project frontend. Create navigation buttons. \\
     - Main vis (team view)          & 25 & Mar. 31  & Implement team view including fetching data, display/layout,  interaction, animation. \\
     - Main vis (student view)       & 15 & Apr. 7   & Implement student view. Some of the team view implementation should be reusable. \\
    Presentation                     & 10 & Apr. 25  & Prepare slides, demo, video(?). Rehearse. \\
    Final paper                      & 20 & Apr. 28  & Finalize paper. Draft to be written Apr. 10-18. \\
    \hline
  \end{tabular*}
\end{table*}

\section{Previous Work}



\bibliographystyle{abbrv}
%\bibliographystyle{abbrv-doi}
%\bibliographystyle{abbrv-doi-narrow}
%\bibliographystyle{abbrv-doi-hyperref}
%\bibliographystyle{abbrv-doi-hyperref-narrow}

\bibliography{proposal}
\end{document}
