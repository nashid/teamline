\documentclass[../manifest.tex]{subfiles}

% - should include scenarios of use and multiple screenshots of your software in action. walk the reader through how your interface succeeds (or acknowledge how it falls short) in solving the intended problem.
% - if you did any evaluation (deployment to target users, computational benchmarks), do report on that here.

\begin{document}

A scenario describing how a TA could use Teamline:

Imagine you are a TA tasked with scaling the final grade of each team member by
the amount they contributed. Upon meeting the team, you open their Teamline to
get a sense of the team dynamics: Did they start early? Did they work consistently?
What was their final grade? This information is immediately available to you
because Teamline defaults to showing the team-view of the most recently due deliverable (Fig. \ref{fig:teaser}).
Within the view, you find that the team made a few early submissions that
increased their grade and then made a large number of submissions very close to
the deadline. From the navigation pane, you notice that only one of the team members
contributed significantly to the final grade.

You then proceed to discuss with each team member individually about their contributions
to the project. The one who made the larger contribution according to Teamline, User1,
had a clear understanding of the code and was able to discuss a couple of challenges
he encountered. At the end of the retrospective, User1 said that each of them had
done the work they agreed to do (which they thought was an even split) but that
his partner started very late which made him apprehensive.
To confirm this, you expand the team-view (Fig. \ref{fig:studentview}) to show individual submissions made by
each team member and do in fact notice that all of the submissions for the other
member, User2, were made the night before the due date and that User1's submissions were
made earlier and more consistently.

Next, you talk to User2. After he explains what he contributed, you mention that by
starting so late, it may negatively impact the group dynamic. User2 denies this by claiming
that he started days earlier but, after you show him Teamline, agrees that he should
start earlier next time. While looking at Teamline, you also notice that User1 did
quite a bit more work for the previous deliverable as well. You point this out to
both User1 and User2 and they are a bit surprised. You help them divide up the work
for the next deliverable more equitably.

Later in the week, you decide to check how the team is proceeding. You immediately
see that several submissions were made. Curious to see if your discussion helped,
you expand the team-view and see that User2 has already made several submissions.
You feel much more confident having scaled back User2's grade by only 20\% since he
is now contributing more.


\end{document}
