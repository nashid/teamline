\documentclass[../manifest.tex]{subfiles}

% - Strengths, weaknesses, limitations (reflect on your approach)
% - Lessons learned (what do you know now that you didn't when you started?)
% - Future work (what would you do if you had more time?)


\begin{document}

\subsection{Limitations of Contribution Metrics}
In section \ref{ssec:data-description} we defined two different ways to measure contribution to a team-based coding project and here we discuss some of their limitations.
The pass rate contribution is an all-or-nothing measure meaning that it only counts code that increases instructor-written tests, neglecting all supporting code. For example, it could be the case that one member writes many lines of code, commits them without passing any new tests, and then their partners makes a small change and passes several tests. In this example, the partner would be given all the credit. However, this can be largely mitigated by ensuring that instructor-written tests are very focused so that small changes to the code would be captured.

Determining contribution to the coverage score can also be challenging due to the fact that small changes to either the student-written test suite or the actual code can cause large changes to the coverage score on which the coverage contribution is based. We partly address this issue by only taking counting changes that cause the coverage score to increase beyond the running maximum.

In both cases, the contribution measures lack robustness and are too dependent on individual metrics. For instance, neither takes into account the developer's skill or the time/effort it took for them to write the code. Also, the strong dependence of the contribution measures on the underlying pass rate and coverage make it hard to generalize to datasets that use different measures of code change and improvement. We discuss some approaches to these problems in the future work section.

% We also defined contribution uniformity to summarize how evenly the contributions were made.

\subsection{Future Work}
- highlight commits that had feedback requested (more linear)






In this paper, we defined an
Teamline is the first attempt at defining contribution in the context of code-based projects so there is



%% Future Work
% - Evaluate our info vis solution with TAs
% - Capture and include more metrics (esp. LOC + code churn) -- generalize the notion of contribution
% - Finer-grain contributions (less black and white -- if a student does 90\% of the code but their partner makes the commit that passes, they will get all the credit)
%
% - Abstract away AutoTest data so that can be used in industry
% - Query live data
% - Can only see one deliverable at a time: no way to see the complete history of the project.
% - Issue with axis





%% Limitations
%  - 100% contribution to coverage but coverage grade only increased a few percent
\end{document}
