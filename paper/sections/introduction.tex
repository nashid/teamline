\documentclass[../manifest.tex]{subfiles}

%% Introduction
% - give the big picture. establish the scope of what you did, some background material may be appropriate here.

\begin{document}
Writing code is a social activity that requires input from all members of a team. Scrum is a popular agile framework that emphasizes this while encouraging awareness of all the work being done by the team. To achieve this, teams run daily Scrum meetings where each developer provides a brief description of what tasks they have done thereby making explicit the contributions of each member.

Scrum works well for co-located teams where it is easy to conduct daily meetings but can be hard to implement when developers are distributed. This is common in open-source projects. Developers typically only interact with each other online so it can be hard for project managers to track individual contributions. GitHub, a popular git hosting service for open-source projects, attempts to address this issue by providing a contributors graph (Fig. \ref{fig:sample-contribution-graph}) but it has very limited functionality.

Our tool, called Teamline\footnote{Demo: https://nickbradley.github.io/teamline}, aspires to provide a visual means of identifying contributions of each team member. To limit the scope of the project we designed Teamline to meet the needs of a single software engineering course at UBC. The course was designed to support a limited version of Scrum but shares many of the challenges of co-located teams and thus provides a reasonably-sized use-case for us to support.

A learning outcome of this course is to develop project management and Scrum skills in the context of a programming project. In particular, students need to develop the ability to determine how equitably they are contributing to the project. To help them learn this, the teaching assistants (TAs) act as a Scrum leader in short meetings where they assess the contributions of each team member. Between the meetings, the students work as a co-located team, using online project management tools to communicate work items with each other.

It is the responsibility of the TAs to scale-back grades of students who did not make a significant contribution before a deliverable deadline. Unfortunately, this can be a hard task for the TA due to both the amount and uniqueness of code written for the project. While Teamline has been designed explicitly for the task of assisting TAs understand contributions made to the project, the framework we developed should generalize to other team-based coding projects.

This paper makes two major contributions.
\begin{itemize}
  \item First, we define several derived attributes that are designed to indicate to what level each member of a team contributed to the overall success of a project. We also give a formula that combines individual contributions to provide a single number indicating the uniformity of contributions.
  \item Second, we provide a prototype visualization that uses the aforementioned contribution attributes to assist users in determining to what extent each member of a team contributed. It uses a heatmap that shows the uniformity of contribution of the users across all cells. This acts as an overview. A detail view shows the contribution of individual commits in side-by-side views, facilitating comparison over time between team members.
\end{itemize}

The remainder of the paper is structured as follows. Section \ref{sec:prev-work} discusses related work. Section \ref{sec:abstractions} defines the domain problem and defines the data attributes and tasks. We present our visualization solution in section \ref{sec:solution} and describe how it supports the tasks. Implementation details for Teamline are presented in section \ref{sec:implementation}. Section \ref{sec:results} presents a scenario that shows how Teamline is intended to be used to support the intended use-case. A discussion of how well Teamline fits the scenario is given in section \ref{sec:discussion} where we also discuss future work. Finally, we conclude in section \ref{sec:conclusion}.

\end{document}
