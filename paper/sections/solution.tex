\documentclass[../manifest.tex]{subfiles}


% - describe your solution idiom, analyze it according to the framework of the book, and justify your design choices with respect to alternative possibilities
% - if you have done any significant algorithmic work, discuss the algorithm and data structures
% - you might choose to split out Interface into its own section

\begin{document}
Our solution is to create a derived quantitative attribute \textit{commitContribution} that describes
the impact of a submission on the overall grade. In particular, the attribute is
the difference between the current submission and the previously graded one. We
visualize this along with other code metrics (Fig. \ref{fig:teaser}) to gain a more
complete understanding of each student's contribution: those who made more grade-improving
submissions should receive a higher retrospective grade. Note that this visualization
is designed to assist the TAs in
making a judgment when assigning a grade and cannot replace them since students may
have chosen a different way to divide the work among team members.



\begin{table}
  \caption{What-Why-How analysis of Teamline.}
  \label{tab:analysis}
  \begin{tabularx}{\columnwidth}{ l | X }
    \hline
    What: Data & Table of graded commits with attributes described in \ref{tab:attributes}. \\
    \hline
    What: Derived & Measure of contribution to pass rate and coverage; contribution uniformity. These are discussed in section \ref{ssec:data-description}. \\
    \hline
    Why: Tasks & Present the uniformity of contributions and summarize the team's commit history. \\
    \hline
    How: Encode & We used a heatmap with sparklines in the overview  and line charts in the detail view. Marks are positioned on a common time scale with color indicating the attribute that is being encoded.\\
    \hline
    How: Facet & Overview + detail views. Detail view is partitioned into side-by-side views. \\
    \hline
    How: Embed & Superimpose sparklines on the overview's heatmap cells. \\
    \hline
    How: Reduce & Filtering is done by selecting the team and the deliverable. \\
    \hline
  \end{tabularx}
\end{table}



% \begin{table}
%   \label{tab:analysis}
%   \caption{What-Why-How analysis of Teamline.}
%   \begin{tabularx}{\columnwidth}{ l | l }
%     \hline
%     What: Data & Table of graded commits. \\
%     What: Derived & \parbox{6cm}{Measure of contribution to pass rate and coverage. Contribution uniformity.}  \\
%     Why: Tasks & \parbox{6cm}{Present the uniformity of contributions and summarize the team's commit history.} \\
%     How: Encode & \parbox{6cm}{We used a heatmap in the overview with}  \\
%     How: Facet & \parbox{6cm}{Overview + detail views. Detail view is partitioned into side-by-side views.} \\
%     How: Embed & \parbox{6cm}{Superimpose sparklines on the overview's heatmap cells.} \\
%     How: Reduce & \parbox{6cm}{Filtering is done by selecting the team and the deliverable in}  \\
%     \hline
%   \end{tabularx}
% \end{table}



% \begin{table}[h!]
%   \centering
%   \caption{Caption for the table.}
%   \label{tab:table1}
%   \begin{tabular}{l|c||r}
%     1 & 2 & 3\\
%     \hline
%     a & b & c\\
%   \end{tabular}
% \end{table}

\end{document}
