\documentclass[../manifest.tex]{subfiles}


% - describe your solution idiom, analyze it according to the framework of the book, and justify your design choices with respect to alternative possibilities
% - if you have done any significant algorithmic work, discuss the algorithm and data structures
% - you might choose to split out Interface into its own section

\begin{document}

Our solution is to create a derived quantitative attribute \textit{commitContribution} that describes
the impact of a submission on the overall grade. In particular, the attribute is
the difference between the current submission and the previously graded one. We
visualize this along with other code metrics (Fig. \ref{fig:teaser}) to gain a more
complete understanding of each student's contribution: those who made more grade-improving
submissions should receive a higher retrospective grade. Note that this visualization
is designed to assist the TAs in
making a judgment when assigning a grade and cannot replace them since students may
have chosen a different way to divide the work among team members.

Here we use the what-why-how framework \cite{Munzner:2014} to abstract our solution to the vis domain.

\textbf{What}. Table of graded submission records (items) with the attributes
described in Table 1. The dataset is static once it has been loaded on the page
but is dynamic in that the dataset grows with each new submission.

\textbf{Why}. Compare the contributions of team members and derive a retrospective
grade for each of them.

\textbf{How}. Our vis includes many encodings and idioms:
\begin{itemize}
  \item Separate individual submissions and align them on two (or three)
        parallel axes by timestamp.
  \item Submissions are represented by area marks. Marks that overlap are collapsed into
    a single point mark whose size encodes the number of submissions that were collapsed.
  \item Marks are coloured with saturation encoding the overall grade for the submission (max saturation for a grade of 100\%).
    We are okay with using a lower effectiveness encoding which may not be visible on marks showing only a single submission
    because the final grade is not the focus of this vis.
  \item Interaction allows details about a submission to be seen:
    \begin{itemize}
      \item Hovering over a single mark displays a popup with detailed information about the submission.
      \item Clicking a single point opens a new browser tab that displays the corresponding commit on GitHub.
      \item Hovering over a grouped mark expands it to show all submissions it includes. Once expanded,
        the above interactions are allowed.
    \end{itemize}
\end{itemize}

\end{document}
