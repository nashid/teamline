\documentclass[../manifest.tex]{subfiles}

% - medium-level implementation description. you must include specifics of what you did yourself versus what other components/libraries/toolkits you built upon. this section is one major divergence from standard research paper format, you need to provide much more detail than would normally be appropriate in a research context. see this good example. teams of two or more people should include a breakdown of who did what work, as in this good example.

\begin{document}
Implementation

The work was divided as: ...
We used the libraries: ...

\subsection{Data Transformation}
Teamline uses the CPSC310 course project data captured in a CouchDB noSQL database during term 2 of the 2016/17 year. We made a copy of this database after the project was complete and deployed the it on a personal server. The database is over 14 GB and contains approximately 44,000 commit records which contain details about the committed code including the pass rate against the instructor-written tests and the percentage of code the team covered with their own tests. These fields are extracted from the database using a map-reduce view whose code is available in \texttt{contribs-by-team.js}. The view emits a list of commit records which can be optionally filtered by team, deliverable and timestamp. On top of this view, we implemented a list function, whose code is available in \texttt{contrib.js}, that takes as input the filtered list of commits and outputs a list of team objects: aggregates of all the commits made by the members of the team. It also computes the contribution metrics discussed in section \ref{ssec:data-description}. A complete description of the fields emitted by the list function can be found in \texttt{data-guide.md}.

% how does it do the aggregate: walks through record-by-record and does the aggregation
% emits a JSON file
% 

\subsection{Overview View}
\subsection{Detail View}

\end{document}
