\documentclass[../manifest.tex]{subfiles}

\begin{document}

Our visualization is designed to show each person's contribution to a team-based project. Previous work by Kelly \cite{Kelly:2016} examined whether visualizing contributions in a team-based, collaborative game leads to fairer contributions by team members. Their approach was based on using a attribute, \textit{meters}, which is derived from existing artifacts that enable awareness of contributions in the game. We use a similarly-purposed derived attribute, \textit{contribution}, that indicates how much each team member contributed to the overall grade. They found that only using this single attribute could undermine the efforts of collaborators if it doesn't adequately "reflect important aspects of individual work in the context of team activity". Also these tools should be combined with other methods to more robustly evaluate contribution in real-time or retrospectively. This finding supports our use-case. Teamline allows team members to see who is contributing to the overall grade of the project as it progresses and helps the TA better understand how the work was divided within the team during the retrospective meeting.

The most critical aspect of our visualization is the ability to easily and accurately compare indicators of contribution over time. Much work has been done exploring effective ways to visualize comparisons between objects. Gleicher, et. al.\cite{Gleicher:2011} give a taxonomy of visual designs used for comparison tasks noting that all designs are assembled using juxtaposition, superposition and explicit encodings (computing the relationships between objects and providing a visual encoding of the relationships). The authors distinguish these categories by the principal mechanism used to make connections between objects: juxtaposition uses the viewer’s memory, superposition uses the visual system and explicit encodings use computation to determine the relationships. These categories can be combined to form hybrid categories. Munzner talks about these approaches in detail in \cite{Munzner:2014}. Teamline takes a hybrid approach by both superimposing each contributors' metrics in the same view and by visually encoding the computed contribution score.

Our vis was inspired by ShiViz\footnote{https://bestchai.bitbucket.io/shiviz/?}\cite{Abrahamson:2014} which shows messages being passed among a collection of processes to verify the happens-before relation is not violated, commit graphs like the one built into BitBucket\footnote{https://marketplace.atlassian.com/plugins/com.plugin.commitgraph.commitgraph/server/overview} which visualize commits in time, the map view on Craigslist\footnote{https://vancouver.craigslist.ca/search/hhh} which shows size-encoded point marks, and the magnifying effect for the OSX dock.


Visualization tools exist that can be used to show contribution to team-based projects that use the git source control system. The most relevant to our specific use-case would be GitHub's \textit{Contributors graph}\footnote{https://help.github.com/articles/viewing-contribution-activity-in-a-repository/}. It uses filled line graphs to show the either the number of commits, number of lines added, or the number of lines deleted over time. There is an aggregate view that shows these metrics for all contributors and there are also individual contribution charts. Users can select a time range in the aggregate view to see the individual contributions only during that time range. This visualization is not sufficient for our task because it only shows the number of commits made by each team member which may not be indicative of their actual contribution to the final grade. Other tools exist that also visualize metrics exposed by git, for instance GitHub Visualizer\footnote{https://help.github.com/articles/viewing-contribution-activity-in-a-repository/}, but share similar shortcomings as the GitHub's Contributors graph.

\end{document}

%
% This describes exactly the use-case for which Teamline was designed. We want to allow team members to be able to see who is contributing to the overall grade of the project and to help the TA better understand how the work was divided between the team during the retrospective meeting.

%
% Our work follows a similar approach in that we use a derived attribute, \textit{contribution}, that serves a similar approach as
%
% is an indicator of how much each team member contributed to the final grade.
%
% While their work focused on collaboration in a online game, we find some similarities to our work on Teamline. In particular, they use \textit{meters} which are existing artifacts that enable awareness of contributions in the game and we use a similar idea with our \textit{contribution} derived attribute. One finding that we take away from their work is that such measures may fail if it fails to combine a set of



% Our vis was inspired by ShiViz\footnote{https://bestchai.bitbucket.io/shiviz/?}\cite{Abrahamson:2014} which shows messages being passed among
% a collection of processes to verify the happens-before relation is not violated,
% commit graphs like the one built into BitBucket\footnote{https://marketplace.atlassian.com/plugins/com.plugin.commitgraph.commitgraph/server/overview}
% which visualize commits in time, the map view on Craigslist\footnote{https://vancouver.craigslist.ca/search/hhh} which shows size-encoded point marks, and the magnifying effect for the OS X dock.
%
% In addition, research around team contribution and collaboration (e.g. \cite{Kelly:2016}) could help us further refine Teamline.


%
% Related:
% https://worvilleanalysis.wordpress.com/2015/09/13/visualizing-and-measuring-defensive-contribution/
% http://dl.acm.org/citation.cfm?doid=2818048.2819977 (Can Visualization of Contributions Support Fairness in Collaboration?: Findings from Meters in an Online Game)
%
